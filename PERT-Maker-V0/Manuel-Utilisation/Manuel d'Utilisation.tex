\documentclass{article}
\usepackage{graphicx}
\usepackage{amsmath,amssymb,enumerate,graphicx,pgf,tikz,fancyhdr}
\usepackage{geometry}
\usepackage{tabvar}
\usepackage{fontspec}
\usepackage{dot2texi}
\usepackage{hyperref}
\usepackage{minted}

\usetikzlibrary{backgrounds}
\usetikzlibrary{arrows.meta}
\usetikzlibrary{shapes.geometric}

\title{\centering PERT Maker: Manuel d'utilisation}

\author{BOYER Timothé, GRASSET Emilien, HACINI Malik}
\date{9 Novembre 2023}
\renewcommand{\contentsname}{Table des Matières}

\renewcommand{\theFancyVerbLine}{
    \sffamily\textcolor[rgb]{0.5,0.5,0.5}{\scriptsize\arabic{FancyVerbLine}}}
    
    
\begin{document}
    
    
\csundef{listing}\csundef{endlisting}
\csundef{listing*}\csundef{endlisting*}

\maketitle
\tableofcontents{}

\section{Introduction}
PERT Maker est une solution logicielle d'utilisation de la méthode PERT de gestion de projet.
La méthode PERT fournit une méthode et des moyens pratiques pour décrire,
représenter, analyser et suivre de manière logique les tâches et le réseau des tâches à réaliser dans le cadre d'une 
action à entreprendre ou à suivre.
Un graphe de dépendances est utilisé. 

Pour chaque tâche, sont indiquées une date de début et de fin au plus tôt et au plus tard. 
Le graphe permet aussi de déterminer le chemin critique qui conditionne la durée minimale du projet.
Le but est de trouver la meilleure organisation possible pour qu'un projet soit terminé dans les délais, et d'identifier les tâches critiques, 
c'est-à-dire les tâches qui ne doivent souffrir d'aucun retard sous peine de retarder l'ensemble du projet.
(\href{https://fr.wikipedia.org/wiki/PERT}{Plus d'informations sur la méthode PERT})

Cependant, tout ceci est fastidieux à faire à la main ! 

Heureusement, PERT Maker est là pour vous !
PERT Maker réalise toutes ces étapes pour vous. 
A partir du fichier CSV représentant l'organisation de votre projet,
nous concevons des fichier .tex comportant toutes les analyses de votre projet par la méthode PERT.

Le PERT est aussi utilisé pour suivre le déroulement d’un projet une fois commencé. Un compte
rendu d'éxécution recense donc, à une date donnée, pour chaque tâche son état d’avancement. Vous pouvez ainsi
reporter les avances et les retards de chaque tâche et, PERT Maker pourra (re)prévoir les nouvelles dates références de votre projet.
Vous devez donc suivre correctement ce manuel pour tirer le plein potentiel de PERT Maker !

\section{Format du fichier Projet}
Le fichier de votre projet doit suivre plusieurs règles strictes pour être correctement analysé par PERT Maker.
\subsection{Le format CSV}
Votre fichier doit être au format CSV (Comma Separated Values)
Il s'agit d'un format de fichier texte, ou les données sont séparées par des virgules.
Il peut ensuite être vu sous forme tabulaire.
Pour générer un fichier CSV, vous pouvez utiliser votre logiciel de tableur préféré :
Microsoft Excel, Google Sheets,LibreOffice, OpenOffice Calc, tous sont capables d'importer
et exporter des fichiers CSV.
Il vous suffit de choisir l'option CSV lors de l'exportation (ou enregistrement) de votre fichier.

\subsection{Comment }
\end{document}