\documentclass{article}
\usepackage{graphicx}
\usepackage{amsmath,amssymb,enumerate,graphicx,pgf,tikz,fancyhdr}
\usepackage{geometry}
\usepackage{tabvar}
\usepackage{fontspec}
\usepackage{dot2texi}

\usepackage{minted}
\usetikzlibrary{backgrounds}
\usetikzlibrary{arrows.meta}
\usetikzlibrary{shapes.geometric}

\title{\centering Mineure Informatique: 
PERT Maker}
\author{BOYER Timothé, GRASSET Emilien, HACINI Malik}
\date{9 Novembre 2023}
\renewcommand{\contentsname}{Table des Matières}

\renewcommand{\theFancyVerbLine}{
    \sffamily\textcolor[rgb]{0.5,0.5,0.5}{\scriptsize\arabic{FancyVerbLine}}}
    
    
    \begin{document}
    
    
    \csundef{listing}\csundef{endlisting}
    \csundef{listing*}\csundef{endlisting*}
    
    \maketitle
    \tableofcontents{}
    
    \section{Introduction}
    
    Ce document relate la conception de PERT Maker.
    PERT Maker est une solution logicielle d'utilisation de la méthode PERT de gestion de projet.
    La méthode PERT fournit une méthode et des moyens pratiques pour décrire,
    représenter, analyser et suivre de manière logique les tâches et le réseau des tâches à réaliser dans le cadre d'une 
    action à entreprendre ou à suivre.
    Un graphe de dépendances est utilisé. 
    
    Pour chaque tâche, sont indiquées une date de début et de fin au plus tôt et au plus tard. 
    Le graphe permet aussi de déterminer le chemin critique qui conditionne la durée minimale du projet.
    Le but est de trouver la meilleure organisation possible pour qu'un projet soit terminé dans les délais, et d'identifier les tâches critiques, 
    c'est-à-dire les tâches qui ne doivent souffrir d'aucun retard sous peine de retarder l'ensemble du projet.

    Ce logiciel est le fruit d'un travail en groupe de longue haleine. Nous avons, pendant près de 2 mois, concu pas à pas le logiciel.
    Nous avons évidemment rencontrés beaucoup de difficultés tout au long de la concpetion, que nous détaillerons.

    Ce rapport s'articule en sections : Conception, 
    
\end{document}