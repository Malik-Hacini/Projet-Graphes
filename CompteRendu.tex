\documentclass{article}
\usepackage{graphicx} % Required for inserting images
\usepackage{pythonhighlight}
\usepackage{amsmath,amssymb,enumerate,graphicx,pgf,tikz,fancyhdr}
\usetikzlibrary{arrows}
\usetikzlibrary{backgrounds}
\usepackage[T1]{fontenc}
\usepackage[french]{babel}
\usepackage{geometry}
\usepackage{tabvar}
\geometry{hmargin=2.2cm,vmargin=1.5cm}

\title{TP 1: Parcours d'Arbre Binaire.}
\author{HACINI Malik}
\date{September 2023}
\renewcommand{\contentsname}{Table des Matières}


\begin{document}

\maketitle
\tableofcontents{}

\section{Introduction}

Le but de ce TP est de représenter un arbre binaire en python via une classe,
puis de le parcourir en profondeur de 3 façons différentes :
\newline
-Préfixe
\newline
-Postfixe
\newline
-Infixe


\section{Classe "Noeud"}
Voici l'implémentation de la classe Noeud.
Chaque noeud a pour attribut l'information qu'il porte (un entier) 
et ses fils gauches et droits, d'autres noeuds.
On ajoute aussi les méthode de classe ajouter-d et ajouter-g, qui
permettent de créer un noeud, fils gauche ou droit d'un autre.

\begin{python}
    from __future__ import annotations

class Noeud:
    def __init__(self,info=int,f_g=None,f_d=None):
        self.info=info
        self.f_g=f_g
        self.f_d=f_
        
    #Ajoute un noeud a l'arbre, en tant que fils de son pere (self)
    def ajouter_d(self,info):
        self.f_d = Noeud(info)
        return self.f_d
    
    def ajouter_g(self,info):
        self.f_g = Noeud(info)
        return self.f_g

\end{python}

\section{Construction de l'Arbre.}

\subsection{Schéma.}
\begin{center}
\resizebox {\textwidth} {!} {

 \begin{tikzpicture}[framed,
    every node/.style = {
    minimum width = 2em, draw, circle},
    level/.style = {sibling distance = 30mm/#1}
    ]
    \node {1}
    child {node {2} 
          child {node {4}
          child{node{7}}
          child{edge from parent[draw = none]}
          child{node{8}}}
          child {edge from parent[draw = none]}
          child {node {5}}
          }
    child {node {3}
           child {edge from parent[draw = none]
           }
           child {edge from parent[draw = none]}
           child {node {6}
                   child {node {9}}
                   child {edge from parent[draw = none]}
                   child {node {10}}}
            };
 \end{tikzpicture}}
\end{center}
\subsection{Implémentation.}
\begin{python}
    #On construit l'arbre de bas en haut.
    n1=Noeud(1)
    n2=n1.ajouter_g(2)
    n3=n1.ajouter_d(3)
    n4=n2.ajouter_g(4)
    n5=n2.ajouter_d(5)
    n6=n3.ajouter_d(6)
    n7=n4.ajouter_g(7)
    n8=n4.ajouter_d(8)
    n9=n6.ajouter_g(9)
    n10=n6.ajouter_d(10)
\end{python}


\section{Parcours de l'Arbre.}
 Nous définissons chaque parcours 
 comme une méthode différente de la classe Noeud.
\subsection{Préfixe.}
ordre : r -> g -> d
\begin{python}
    def prefixe(self)->list:
        info=[self.info]
        if self.f_g!= None:
            info= info + self.f_g.prefixe()
        if self.f_d!= None:
            info= info + self.f_d.prefixe() 
        return info
\end{python}
\subsection{Postfixe.}
Ordre : g d r
\begin{python}
    def postfixe(self)->list:
    info=[]
    if self.f_g!=None:
        info= info + self.f_g.postfixe()  
        
    if self.f_d!=None:
        info= info + self.f_d.postfixe()
    info = info + [self.info]
    return info
\end{python}

\subsection{Infixe.}
ordre : g r d
\begin{python}
    def infixe(self)->list:
    info=[]
    if self.f_g!=None:
        info= info + self.f_g.infixe() 
         
    info = info + [self.info]
    
    if self.f_d!=None:
        info= info + self.f_d.infixe()
    return info
\end{python}
\section{Tests.}
On teste avec pytest des noeuds de tout types, sur chaque parcours.
\subsection{Préfixe.}

\begin{python}
    def test_prefixe():
    
    assert n1.prefixe()==[1, 2, 4, 7, 8, 5, 3, 6, 9, 10]
    assert n4.prefixe()==[4, 7, 8]
    assert n5.prefixe()==[5]
    assert n9.prefixe()==[9]
\end{python}

\subsection{Postfixe.}

\begin{python}
    def test_postfixe():
assert n1.postfixe()==[7, 8, 4, 5, 2, 9, 10, 6, 3, 1]
assert n4.postfixe()==[7, 8, 4]
assert n5.postfixe()==[5]
assert n9.postfixe()==[9]
\end{python}

\subsection{Infixe.}

\begin{python}
    def test_infixe():
    assert n1.infixe()==[7, 4, 8, 2, 5, 1, 3, 9, 6, 10]
    assert n4.infixe()==[7, 4, 8]
    assert n5.infixe()==[5]
    assert n9.infixe()==[9]
\end{python}
\subsection{Résultats}
On lance pytest : 
Les tests sont tous réussis YAHOUU c fini.
\end{document} 
