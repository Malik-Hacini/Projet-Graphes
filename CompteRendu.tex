\documentclass{article}
\usepackage{graphicx} % Required for inserting images
\usepackage{pythonhighlight}
\usepackage{amsmath,amssymb,enumerate,graphicx,pgf,tikz,fancyhdr}
%\usepackage[utf8]{inputenc}
\usetikzlibrary{arrows}
\usepackage[T1]{fontenc}
\usepackage[french]{babel}
\usepackage{geometry}
\usepackage{tabvar}
\geometry{hmargin=2.2cm,vmargin=1.5cm}

\title{TP 1: Parcours d'Arbre Binaire.}
\author{HACINI Malik}
\date{September 2023}
\renewcommand{\contentsname}{Table des Matières}


\begin{document}

\maketitle
\tableofcontents{}

\section{Introduction}

Le but de ce TP est de représenter un arbre binaire en python via une classe,
puis de le parcourir en profondeur de 3 façons différentes :
\newline
-Préfixe
\newline
-Postfixe
\newline
-Infixe


\section{Classe "Noeud"}
Voici l'implémentation de la classe Noeud.
Chaque noeud a pour attribut l'information qu'il porte (un entier) 
et ses fils gauches et droits, d'autres noeuds.
On ajoute aussi les méthode de classe ajouter-d et ajouter-g, qui
permettent de créer un noeud, fils gauche ou droit d'un autre.

\begin{python}
    from __future__ import annotations

class Noeud:
    def __init__(self,info,f_g=None,f_d=None):
        self.info=info
        self.f_g=f_g
        self.f_d=f_
        
    #Ajoute un noeud a l'arbre, en tant que fils de son pere (self)
    def ajouter_d(self,info):
        self.f_d = Noeud(info,self)
        return self.f_d
    
    def ajouter_g(self,info):
        self.f_g = Noeud(info,self)
        return self.f_g

\end{python}

\section{Construction de l'Arbre}

\begin{center}
\resizebox {\textwidth} {!} {

 \begin{tikzpicture}[
    every node/.style = {minimum width = 2em, draw, circle},
    level/.style = {sibling distance = 30mm/#1}
    ]
    \node {1}
    child {node {2} 
          child {node {4}
          child{node{7}}
          child{edge from parent[draw = none]}
          child{node{8}}}
          child {edge from parent[draw = none]}
          child {node {5}}
          }
    child {node {3}
           child {edge from parent[draw = none]
           }
           child {edge from parent[draw = none]}
           child {node {6}
                   child {node {9}}
                   child {edge from parent[draw = none]}
                   child {node {10}}}
            };
  \end{tikzpicture}}
\end{center}

\end{document} 