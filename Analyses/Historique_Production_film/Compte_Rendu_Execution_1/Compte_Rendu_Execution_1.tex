\PassOptionsToPackage{dvipsnames}{xcolor}
\documentclass{article}
\usepackage{graphicx}
\usepackage{amsmath,amssymb,enumerate,graphicx,pgf,tikz,fancyhdr}
\usepackage[dvipsnames]{xcolor}
\usepackage{geometry}
\usepackage{tabvar}
\usepackage{dot2texi}
\usetikzlibrary{backgrounds}
\usetikzlibrary{arrows.meta}
\usetikzlibrary{shapes.geometric}
\title{\centering Peer Review and Evaluation Technique : Production film
}

\author{PERT Maker}
\renewcommand{\contentsname}{Table des Matières}
\begin{document}
\maketitle
\tableofcontents{}
\section{Votre Graphe de tâches}
\begin{center}
\begin{tikzpicture}[scale=0.6, every node/.style={scale=0.6}]
\begin{dot2tex}[codeonly]

    digraph G{ 
   rankdir=LR 
   S -> A [label="0"]; 
   S -> M [label="0"]; 
   S -> B [label="0" color="red"]; 
   A -> B [label="0"]; 
   M -> D [label="0"]; 
   D -> B [label="0"]; 
   B -> MO [label="60" color="red"]; 
   B -> P [label="60"]; 
   MO -> F [label="30" color="red"]; 
   P -> F [label="14"]; 
   } 
\end{dot2tex}
\end{tikzpicture}
\end{center}
\section{Analyse de votre projet}
Votre projet possède un temps incompressible de 93 jours.
    Il s'agit de la durée totale du \textcolor{red}{chemin critique}.
    Pour terminer le projet dans ces délais, les \textcolor{red}{tâches critiques} ne doivent prendre aucun retard.\subsection{Dates de référence pour chaque tâche}Chaque tâche du chemin critique ne doit pas prendre de
    retard, car cela entrainerait un retard global du projet. 
    Chaque tâche possède trois dates de référence : La date de début au plus tôt,
    fin au plus tôt, fin au plus tard.
    Attention, chaque tâche du chemin critique ne doit pas prendre de
    retard, car cela entrainerait un retard global du projet.
    Leurs dates de début au plus tôt seront donc marquées en rouge,
    car indispensables à respecter. La tâche de départ n'est pas incluse. \newpage
    Voici le tableau récapitulatif des dates de référence pour votre projet :\newline 
\begin{tabular}{ |p{3cm}||p{3cm}|p{3cm}|p{3cm}|  }
        \hline
        \multicolumn{4}{|c|}{Dates de références} \\
        \hline 
        Tâche&Début au plus tôt&Fin au plus tôt&Fin au plus tard \\ 
        \hline 
S&T0&T0+0&T0+0 \\ 
A&T0+0&T0+0&T0+0 \\ 
M&T0+0&T0+0&T0+0 \\ 
D&T0+0&T0+0&T0+0 \\ 
\textcolor{red}{B}&T0+\textcolor{red}{0}&T0+60&T0+60 \\ 
\textcolor{red}{MO}&T0+\textcolor{red}{60}&T0+90&T0+90 \\ 
P&T0+60&T0+74&T0+90 \\ 
F&T0+90&T0+93&T0+93 \\ 
\hline
    \end{tabular} 
\end{document}