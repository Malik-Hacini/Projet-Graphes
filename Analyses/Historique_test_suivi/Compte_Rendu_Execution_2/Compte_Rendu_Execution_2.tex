\PassOptionsToPackage{dvipsnames}{xcolor}
\documentclass{article}
\usepackage{graphicx}
\usepackage{amsmath,amssymb,enumerate,graphicx,pgf,tikz,fancyhdr}
\usepackage[dvipsnames]{xcolor}
\usepackage{geometry}
\usepackage{tabvar}
\usepackage{dot2texi}
\usetikzlibrary{backgrounds}
\usetikzlibrary{arrows.meta}
\usetikzlibrary{shapes.geometric}
\title{\centering Peer Review and Evaluation Technique : Votre Projet. 
}

\author{PERT Maker}
\renewcommand{\contentsname}{Table des Matières}
\begin{document}
\maketitle
\tableofcontents{}
\section{Votre Graphe de tâches}
\begin{center}
\begin{tikzpicture}[scale=0.6, every node/.style={scale=0.6}]
\begin{dot2tex}[codeonly]

    digraph G{ 
   rankdir=LR 
   D -> I3 [label="60"]; 
   D -> I2 [label="60"]; 
   D -> I1 [label="60" color="red"]; 
   I1 -> F [label="5" color="red"]; 
   I2 -> F [label="90"]; 
   I3 -> F [label="7"]; 
   } 
\end{dot2tex}
\end{tikzpicture}
\end{center}
\section{Analyse de votre projet}
Votre projet possède un temps incompressible de 125 jours.
    Il s'agit de la durée totale du \textcolor{red}{chemin critique}.
    Pour terminer le projet dans ces délais, les durées de toutes les tâches parallèles au tâches critiques doivent pouvoir être réduites
    à la même durée que celle des tâches critiques, et aucune tâche ne doit prendre du retard.\subsection{Dates de référence pour chaque tâche}Commencons par le plus important. Chaque tâche du chemin critique ne doit pas prendre de
    retard, car cela entrainerait un retard global du projet.
    Chaque tâche possède trois dates de référence : La date de début au plus tôt,
    fin au plus tôt, fin au plus tard.
    Voici le tableau récapitulatif des dates de référence pour votre projet :\newline 
\begin{tabular}{ |p{3cm}||p{3cm}|p{3cm}|p{3cm}|  }
        \hline
        \multicolumn{4}{|c|}{Dates de références} \\
        \hline 
        Tâche&Début au plus tôt&Fin au plus tôt&Fin au plus tard \\ 
        \hline 
 D&T0+0&T0+60&T0+60 \\ 
 I1&T0+60&T0+65&T0+65 \\ 
 I2&T0+60&T0+150&T0+150 \\ 
 I3&T0+60&T0+67&T0+67 \\ 
 F&T0+65&T0+125&T0+210 \\ 
\hline
    \end{tabular} 
\end{document}